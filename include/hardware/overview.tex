\subsection{Overview}
Our current setup consists of one master and one slave GBTx board.
The master is connected to the MiniDAQ GBTx channel 3 (fiber 8),
and the slave can be connected to either GBT channel 0 (fiber 6) or channel 6
(fiber 11).
The master synchronizes its on-board clock to the signal from the MiniDAQ,
and propagates such clock signal to the slave.
The slave does not have an on-board clock,
and is configured to obtain clock signal externally.

The master \itwoc port is connected to an external USB device.
The slave \itwoc port is connected to the master.
Both are set to be programmed by the \itwoc channel,
rather than GBT-IC channel.

The current setup is capable of:
\begin{enumerate}
    \item Read/Write the register value of the master GBTx board with GBT-IC
        specification on the MiniDAQ.
    \item Program the slave GBTx board with MiniDAQ directly.
    \item Do a loop back test from MiniDAQ to the slave, then back to the
        MiniDAQ.
        The master is also required\footnote{
            The master GBTx is also connected to the MiniDAQ with a different
        channel.}
        as the slave can only obtain its reference clock from the master.
\end{enumerate}

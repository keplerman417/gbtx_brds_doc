\subsubsection{Overview}

GBTx daughter board (GBTx DB) is a prototype GBTx board, built to verify GBTx
ASIC functionalities for LHCb.
It is a very flexible board that can take many configurations.

Our current setup consists of one master and one slave GBTx DB.
When the slave is in widebus transceiver mode,
the master is connected to the MiniDAQ GBT channel 3 (fiber 8),
and the slave can be connected to either GBT channel 0 (fiber 6) or channel 6
(fiber 11).
The master synchronizes its on-board clock to the signal from the MiniDAQ,
and propagates its clock signal to the slave.
The slave does not have an on-board clock,
and is configured to obtain clock signal externally.

When the slave is in widebus transmitter mode,
again the master is connected to channel 3,
and the slave can be connected to any of the MiniDAQ channel 6-11
(physical fiber 11, 9, \ldots, 1).
We use these channels to connect to DCB data GBTxs.

The master \itwoc port is connected to an external USB device.
The slave \itwoc port is connected to the master.
Both are set to be programmed by the \itwoc channel,
rather than GBT-IC channel.

The current setup is capable of:
\begin{enumerate}
    \item Programming the slave GBTx DB with MiniDAQ directly.
    \item Read/Write the register value of the master GBTx DB with GBT-IC
        specification on the MiniDAQ.
    \item Perform PRBS tests from MiniDAQ to the slave, with slave configured in
        either transceiver or transmitter mode, then back to the MiniDAQ.
        The master is also required\footnote{
            The master GBTx is also connected to the MiniDAQ with a different
        channel, to provide reference clock to the slave.}
        as the slave can only obtain its reference clock from the master.
\end{enumerate}
